\documentclass[12pt]{report}
\usepackage[UTF8]{ctex}
\usepackage[a4paper]{geometry}
\usepackage[myheadings]{fullpage}
\usepackage{fancyhdr}
\usepackage{lastpage}
\usepackage{graphicx, wrapfig, subcaption, setspace, booktabs}
\usepackage[T1]{fontenc}
\usepackage[font=small, labelfont=bf]{caption}
\usepackage{fourier}
\usepackage[protrusion=true, expansion=true]{microtype}
\usepackage[english]{babel}
\usepackage{sectsty}
\usepackage[colorlinks,linkcolor=black,urlcolor=blue]{hyperref}
\usepackage{graphicx}

\usepackage{ulem}

\usepackage{xcolor}
\usepackage{listings}

\usepackage{setspace}
\usepackage{indentfirst}
\setlength{\parindent}{2em}


\newcommand{\HRule}[1]{\rule{\linewidth}{#1}}
\renewcommand\thesection{\arabic{section}}
\setcounter{tocdepth}{5}
\setcounter{secnumdepth}{5}


\definecolor{dkgreen}{rgb}{0,0.6,0}
\definecolor{gray}{rgb}{0.5,0.5,0.5}
\definecolor{mauve}{rgb}{0.58,0,0.82}

\lstset{ %
  language=Java,                % the language of the code
  basicstyle=\footnotesize,           % the size of the fonts that are used for the code
  numbers=left,                   % where to put the line-numbers
  numberstyle=\tiny\color{gray},  % the style that is used for the line-numbers
  stepnumber=2,                   % the step between two line-numbers. If it's 1, each line 
                                  % will be numbered
  numbersep=5pt,                  % how far the line-numbers are from the code
  backgroundcolor=\color{white},      % choose the background color. You must add \usepackage{color}
  showspaces=false,               % show spaces adding particular underscores
  showstringspaces=false,         % underline spaces within strings
  showtabs=false,                 % show tabs within strings adding particular underscores
  frame=single,                   % adds a frame around the code
  rulecolor=\color{black},        % if not set, the frame-color may be changed on line-breaks within not-black text (e.g. commens (green here))
  tabsize=2,                      % sets default tabsize to 2 spaces
  captionpos=b,                   % sets the caption-position to bottom
  breaklines=true,                % sets automatic line breaking
  breakatwhitespace=false,        % sets if automatic breaks should only happen at whitespace
  title=\lstname,                   % show the filename of files included with \lstinputlisting;
                                  % also try caption instead of title
  keywordstyle=\color{blue},          % keyword style
  commentstyle=\color{dkgreen},       % comment style
  stringstyle=\color{mauve},         % string literal style
  escapeinside={\%*}{*)},            % if you want to add LaTeX within your code
  morekeywords={*,...}               % if you want to add more keywords to the set
}


%-------------------------------------------------------------------------------
% HEADER & FOOTER
%-------------------------------------------------------------------------------
\pagestyle{fancy}
\fancyhf{}
\setlength\headheight{15pt}
\fancyfoot[R]{Page \thepage\ of \pageref{LastPage}}
%-------------------------------------------------------------------------------
% TITLE PAGE
%-------------------------------------------------------------------------------

\begin{document}

\title{ \normalsize \textsc{Report}
        \\ [2.0cm]
        \HRule{0.5pt} \\
        \LARGE \textbf{\uppercase{基于智能合约的分布式大富翁游戏}}
        \HRule{2pt} \\ [0.5cm]
        \normalsize \today \vspace*{5\baselineskip}}

\date{}

\author{
        姓名: 杨晨宇、蓝宇霆 \\
        学号: 517030910386、517030910368 }


\maketitle
\tableofcontents
\newpage

\sectionfont{\scshape}
%-------------------------------------------------------------------------------

%-------------------------------------------------------------------------------
% BODY
%-------------------------------------------------------------------------------

\section{项目介绍}
\subsection{背景}
\subsection{相关工作}
在实现该项目的过程中,我们主要参考了Truffle官方文档\cite{1}和Web3.js官方文档\cite{2}。此外,由于编写合约的Solidity及与合约的交互过程对我们而言都较为陌生,因此我们也阅读了许多开源项目的源代码,其中包括官方样例Pet-Shop\cite{3}以及非常有名的入门项目僵尸工厂\cite{4}等等。虽然这些项目都与我们要做的大富翁有一定区别,但通过它们我们对于智能合约的应用有了更深刻的了解。

此外,我们还参考了github上一个利用智能合约实现五子棋的项目\cite{5}。该项目与我们的目的较为接近——事实上,我们的房间创建/加入的想法就脱胎于此,并在其基础上从双人扩展到多人。但该项目版本过于古老,且游戏机制远不如大富翁复杂,因此也只是作为参考,并未用于实际实现中。

我们的主体框架是在官方样例webpack-box\cite{6}的基础上搭建的。该项目原本仅用来展示虚拟货币的支付过程。我们采用了它的目录结构,而其他部分,包括前端设计,页面交互,事件广播及合约逻辑最终全都由我们自己实现。

\section{项目环境}
truffle web3.js webpack solidity WebStorm IDE Remix IDE
\section{项目实现}
\subsection{合约实现}
\subsubsection{数据结构}
\subsubsection{合约事件}
\subsubsection{主要函数}
\begin{lstlisting}
function createRoom(uint32 _roomId) public payable ;
\end{lstlisting}
\subsection{前端交互}
。。。。
在创建房间/加入房间以后,根据当前房间人数,玩家会被分配一个座位号。考虑到实际测试是在私链上进行,我们利用房间号+座位号的方式唯一确定一名玩家的身份,并利用他的地址作为验证。
\section{项目难点}

\section{后续工作}
\section{分工细节}
\section{参考资料}
\begin{enumerate}
	\bibitem{1} Truffle官网 \url{https://www.trufflesuite.com/docs}
	\bibitem{2}	Web3.js官网 \url{https://web3js.readthedocs.io/en/v1.2.9/}
	\bibitem{3} 官方样例Pet-Shop \url{https://github.com/truffle-box/pet-shop-box}
	\bibitem{4} 僵尸工厂 \url{https://cryptozombies.io/en/course}
	\bibitem{5} 利用智能合约实现五子棋对战 \url{https://github.com/Alexygui/Gobang}
	\bibitem{6} WebPack \url{https://github.com/truffle-box/webpack-box}
\end{enumerate}




\section{完整代码}
\noindent\url{https://github.com/Achronferry/SJTU-EE357-project-BlockChain}


\end{document}

